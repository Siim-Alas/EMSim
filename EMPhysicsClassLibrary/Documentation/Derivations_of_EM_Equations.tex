\documentclass{article}
\usepackage[utf8]{inputenc}
\usepackage{amsmath}
\usepackage{amssymb}

\usepackage{hyperref}
\hypersetup{
    colorlinks=true,
    linkcolor=blue,
    filecolor=magenta,      
    urlcolor=cyan,
}

\begin{document}

\section{Derivations}

This file includes the derivations of the equations used in computing various quantities of the \href{https://en.wikipedia.org/wiki/Electromagnetic_field}{electromagnetic field}.

\subsection{Retarded potentials}

This section includes the derivations of the equations used to compute the \href{https://en.wikipedia.org/wiki/Retarded_potential}{retarded potentials}, defined in the Wikipedia article as

\begin{equation}
    \label{eq:definitionOfPhi}
    \phi(\vec{r}, t) = \frac{1}{4 \pi \epsilon_0} \int \frac{\rho(\vec{r'}, t_r)}{|\vec{r} - \vec{r'}|} d \vec{r'}
\end{equation}
\begin{equation}
    \label{eq:definitionOfA}
    \vec{A}(\vec{r}, t) = \frac{\mu_0}{4 \pi} \int \frac{\vec{J}(\vec{r'}, t_r)}{|\vec{r} - \vec{r'}|} d \vec{r'}
\end{equation}

where $\phi(\vec{r}, t)$ is the retarded \href{https://en.wikipedia.org/wiki/Electric_potential}{electric potential}, $\vec{A}(\vec{r}, t)$ is the retarded \href{https://en.wikipedia.org/wiki/Magnetic_vector_potential}{magnetic vector potential}, $\rho(\vec{r'}, t)$ is the \href{https://en.wikipedia.org/wiki/Charge_density}{charge density}, $\vec{J}(\vec{r'}, t_r)$ is the \href{https://en.wikipedia.org/wiki/Current_density}{current density}, and $t_r = t - \frac{|\vec{r} - \vec{r'}|}{c}$ is the \href{https://en.wikipedia.org/wiki/Retarded_time}{retarded time}.

\subsubsection{The effect of a time-invariant point charge on $\phi(\vec{r}, t)$}

The time-invariant point charge is modelled as having \href{https://en.wikipedia.org/wiki/Charge_density}{charge density}

\begin{equation}
    \label{eq:chargeDensityOfAPointCharge}
    \rho(\vec{r}, t) = q \delta(\vec{r} - \vec{r_c})
\end{equation}

where $q$ is the \href{https://en.wikipedia.org/wiki/Electric_charge}{electric charge}, $\vec{r_c}$ is the position vector of the point charge, $\delta(\vec{x})$ is the \href{https://en.wikipedia.org/wiki/Dirac_delta_function}{Dirac delta function}, generalized in the Wikipedia article to 
multiple dimensions via the identity

\begin{equation}
    \label{eq:DiracDeltaIdentityInMultipleDimensions}
    \int_{\mathbb{R}^n} f(\vec{x}) \delta(\vec{x}) d\vec{x} = f(\vec{0})
\end{equation}

which allows us to rewrite equation \ref{eq:definitionOfPhi} as

\begin{equation}
    \label{eq:phiOfAPointCharge}
    \phi(\vec{r}, t) = \frac{1}{4 \pi \epsilon_0} \int \frac{q \delta(\vec{r'} - \vec{r_c})}{|\vec{r} - \vec{r'}|} d \vec{r'} = \frac{1}{4 \pi \epsilon_0} \frac{q}{|\vec{r} - \vec{r_c}|}
\end{equation}

meaning that, because \href{https://en.wikipedia.org/wiki/Integral}{integration} is linear, the effect of a group of point charges on $\phi(\vec{r}, t)$ can be modeled as sum of such components.

\subsubsection{The effect of a time-invariant point charge on $\nabla \phi(\vec{r}, t)$}

Using equation \ref{eq:phiOfAPointCharge}, the effect a time-invariant point charge has on the \href{https://en.wikipedia.org/wiki/Gradient}{gradient} of $\phi(\vec{r}, t)$ is

\begin{equation}
    \label{eq:gradientOfPhiOfAPointCharge}
    \nabla \phi(\vec{r}, t) = \nabla \left( \frac{1}{4 \pi \epsilon_0} \frac{q}{|\vec{r} - \vec{r_c}|} \right) = \frac{q}{4 \pi \epsilon_0} \nabla \left( \frac{1}{|\vec{r} - \vec{r_c}|} \right) = \frac{q}{4 \pi \epsilon_0} \frac{\vec{r_c} - \vec{r}}{|\vec{r} - \vec{r_c}|^3}
\end{equation}

\subsubsection{The effect of a straight 'wire' on $\vec{A}(\vec{r}, t)$}

A straight 'wire' is modelled as a \href{https://en.wikipedia.org/wiki/Line_segment}{line segment} with unit \href{https://en.wikipedia.org/wiki/Tangent_vector}{tangent vector} $\hat{v}$ and a \href{https://en.wikipedia.org/wiki/Current_density}{current density}, which is $\vec{J}(\vec{r'}, t_r) \parallel \hat{v}$ on the line segment and $\vec{0}$ everywhere else.

For convenience, equation \ref{eq:definitionOfA} is repeated here:

\begin{equation*}
    \vec{A}(\vec{r}, t) = \frac{\mu_0}{4 \pi} \int \frac{\vec{J}(\vec{r'}, t_r)}{|\vec{r} - \vec{r'}|} d \vec{r'}
    \tag{\ref{eq:definitionOfA}}
\end{equation*}

Changing to \href{https://en.wikipedia.org/wiki/Translation_(geometry)}{translated} \href{https://en.wikipedia.org/wiki/Spherical_coordinate_system}{spherical coordinates} via the \href{https://en.wikipedia.org/wiki/Coordinate_system#Transformations}{transformation}

\begin{equation}
    \label{eq:translatedSphericalCoordinatesForStraightWire}
    \vec{r'} = \vec{r_0} + \rho  
    \begin{bmatrix}
    \sin(\varphi) \cos(\theta) \\
    \sin(\varphi) \sin(\theta) \\
    \cos(\varphi)
    \end{bmatrix}
\end{equation}

and picking

\begin{equation}
    \label{eq:JForStraightWire}
    \vec{J}(\vec{r'}, t_r) = \frac{\delta(\varphi - \varphi_0) \delta(\theta - \theta_0) f(t_r) g(\rho)}{\rho^2 \sin(\varphi)} \hat{v}
\end{equation}

where

\begin{equation}
    \label{eq:vForStraightWire}
    \hat{v} = 
    \begin{bmatrix}
    \sin(\varphi_0) \cos(\theta_0) \\
    \sin(\varphi_0) \sin(\theta_0) \\
    \cos(\varphi_0)
    \end{bmatrix}
\end{equation}

we get

\begin{equation*}
    \vec{A}(\vec{r}, t) = \frac{\mu_0}{4 \pi} \int \frac{\vec{J}(\vec{r'}, t_r)}{|\vec{r} - \vec{r'}|} d \vec{r'} = \frac{\mu_0 \hat{v}}{4 \pi} \iiint \frac{\delta(\varphi - \varphi_0) \delta(\theta - \theta_0) f(t_r) g(\rho)}{|\vec{r} - \vec{r'}|} d \rho d \varphi d \theta
\end{equation*}

As the $\delta(\varphi - \varphi_0) \delta(\theta - \theta_0)$ in the \href{https://en.wikipedia.org/wiki/Fraction}{numerator} of the \href{https://en.wikipedia.org/wiki/Integral#Terminology_and_notation}{integrand} guarantees it will be $0$ for all $\varphi \neq \varphi_0$ and $\theta \neq \theta_0$, we can safely expand the \href{https://en.wikipedia.org/wiki/Integral}{domain of integration} from $\varphi \in [0; \pi]; \theta \in [0; 2 \pi)$ to $\varphi, \theta \in \mathbb{R}$, as the added members in the 'sum' will all be $0$.

Using equation \ref{eq:DiracDeltaIdentityInMultipleDimensions}, we can simplify the integral to

\begin{equation*}
    \frac{\mu_0 \hat{v}}{4 \pi} \int \frac{f(t - \frac{|\vec{r} - \rho \hat{v}|}{c}) g(\rho)}{|\vec{r} - \rho \hat{v}|} d \rho
\end{equation*}

\end{document}
