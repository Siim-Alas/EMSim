\documentclass{article}
\usepackage[utf8]{inputenc}

\usepackage{hyperref}
\hypersetup{
    colorlinks=true,
    linkcolor=blue,
    filecolor=magenta,      
    urlcolor=cyan,
}

\begin{document}

\section{Derivations}

This file includes the derivations of the equations used in computing various quantities of the \href{https://en.wikipedia.org/wiki/Electromagnetic_field}{electromagnetic field}.

\subsection{Retarded potentials}

This section includes the derivations of the equations used to compute the \href{https://en.wikipedia.org/wiki/Retarded_potential}{retarded potentials}, defined in the Wikipedia article as

$$\phi(\vec{r}, t) = \frac{1}{4 \pi \epsilon_0} \int \frac{\rho(\vec{r'}, t_r)}{|\vec{r} - \vec{r'}|} d \vec{r'}$$
$$\vec{A}(\vec{r}, t) = \frac{\mu_0}{4 \pi} \int \frac{\vec{J}(\vec{r'}, t_r)}{|\vec{r} - \vec{r'}|} d \vec{r'}$$

where $\phi(\vec{r}, t)$ is the retarded \href{https://en.wikipedia.org/wiki/Electric_potential}{electric potential}, $\vec{A}(\vec{r}, t)$ is the retarded \href{https://en.wikipedia.org/wiki/Magnetic_vector_potential}{magnetic vector potential}, $\rho(\vec{r'}, t)$ is the \href{https://en.wikipedia.org/wiki/Charge_density}{charge density}, $\vec{J}(\vec{r'}, t_r)$ is the \href{https://en.wikipedia.org/wiki/Current_density}{current density}, and $t_r = t - \frac{|\vec{r} - \vec{r'}|}{c}$ is the \href{https://en.wikipedia.org/wiki/Retarded_time}{retarded time}.

\subsubsection{The effect of a time-invariant point charge on $\phi(\vec{r}, t)$}

The time-invariant point charge is modeled as having \href{https://en.wikipedia.org/wiki/Charge_density}{charge density}

$$\rho(\vec{r}, t) = q \delta(\vec{r} - \vec{r_c})$$

where $q$ is the \href{https://en.wikipedia.org/wiki/Electric_charge}{electric charge}, $\vec{r_c}$ is the position vector of the point charge, $\delta(\vec{x})$ is the \href{https://en.wikipedia.org/wiki/Dirac_delta_function}{Dirac delta function}, generalized in the Wikipedia article to 
multiple dimensions via the identity

$$\int_{R^n} f(\vec{x}) \delta(\vec{x}) d\vec{x} = f(\vec{0})$$

which allows us to rewrite the equation for the retarded \href{https://en.wikipedia.org/wiki/Electric_potential}{electric potential} as

$$\phi(\vec{r}, t) = \frac{1}{4 \pi \epsilon_0} \frac{q}{|\vec{r} - \vec{r_c}|}$$

meaning that, because \href{https://en.wikipedia.org/wiki/Integral}{integration} is linear, the effect of a group of point charges on $\phi(\vec{r}, t)$ can be modeled as sum of such components.

\subsubsection{The effect of a time-invariant point charge on $\nabla \phi(\vec{r}, t)$}

Using the result of the previous section, the effect a time-invariant point charge has on the \href{https://en.wikipedia.org/wiki/Gradient}{gradient} of $\phi(\vec{r}, t)$ is

$$\nabla \phi(\vec{r}, t) = \nabla \left( \frac{1}{4 \pi \epsilon_0} \frac{q}{|\vec{r} - \vec{r_c}|} \right) = \frac{q}{4 \pi \epsilon_0} \nabla \left( \frac{1}{|\vec{r} - \vec{r_c}|} \right) = \frac{q}{4 \pi \epsilon_0} \frac{\vec{r_c} - \vec{r}}{|\vec{r} - \vec{r_c}|^3}$$

\end{document}
